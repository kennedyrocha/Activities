\documentclass[a4paper,12pt]{article}

%	-----------------------------------------------------------------------------	%
%	---------------------------	Pacotes Utilizados	-----------------------------	%
%	-----------------------------------------------------------------------------	%

%	-----------------------------------------------------------------------------	%
%	-------------------------------	Definições	---------------------------------	%
%	-----------------------------------------------------------------------------	%

\setlength{\parindent}{1.3cm} 				% Espaçamento dos paragrafos
\setlength{\parskip}{.2cm} 					% Distância entre paragrafos
\setlength{\oddsidemargin}{0.5cm}   		% Há um offset obrigatorio de
\setlength{\evensidemargin}{0.0cm}   		% 1 inch do lado esquerdo e no
\setlength{\topmargin}{-1.2cm}       		% topo da folha
\setlength{\headsep}{1.0cm}
\setlength{\textwidth}{15.5cm}
\setlength{\textheight}{24.2cm}
\renewcommand{\baselinestretch}{1.2}
\renewcommand{\labelitemi}{\tiny{\textbullet}}
\begin{document}

\begin{center}
\textbf{{\LARGE Introdução}} \\ \vspace{0.5cm}
\end{center}

\textit{Data Center} é um conjunto de recursos físicos (servidores, redes e \textit{storages} e etc.) dedicados que proporciona um alto poder computacional as telecomunicações, processamento, armazenamento e disponibilização de informações. Tais recursos requerem um alto investimento para serem implantados e mantidos, logo é necessário a utilizão e otimização ao máximo dos dispositivos adiquiridos.

O \textit{data center} é composto de servidores distribuídos em \textit{racks} e a comunicação entre eles ocorre através de cabos e \textit{switches}(que gerência o tráfego de informações entre os dispositivos), que são dispostos num devido local cuja a infraestrutura suporta o funcionamento de todos os equipamentos provendo segurança, energia e controle de temperatura. O objetivo desses recursos estarem num mesmo espaço é trabalhar juntos para aumentar o poder computacional, sendo que cada componente dentro do mesmo pode ser reconhecido como servidor e/ou Sistema em Chip(\textit{System-on-a-Chip}(SoC)).

O SoC é o conjunto de componentes como memória RAM, entrada e saída de dados, disco dígido, processador e etc., num computador ou outro sistema eletrônico e que contenha uma determinada rotina e geralmente um Sistema Operacional(SO) para controlar as mesmas. Os SoCs não estão limitados a computadores e estão presentes em outros disposítivos como smartphones, tablets, mouses e microcontroladores. Quando há mais de um processador num SoC, denomina-se Sistema em Chip com Multiprocessador(\textit{Multi-processor System-on-a-Chip}(MPSoC)), onde há controle de entrada e saída de dados, distribuíção dos processos pelo SO para os processadores do MPSoC. Também há admistração de acesso a memória, que acontece de maneira compartilhada ou distribuída, ou seja, na memória compartilhada vários processadores acessam o mesma memória, enquanto a distribuída cada um possui um bloco de memória específico. Esses processadores se comunicam através de um Sistema de Intercomunicação(SI) chamado de Rede em Chip(Network-on-Chip(NoC)), sendo que o diálogo entre esses eles para processar algo pode ser tanto entre os do mesmo SoC(Intra-Chip), quanto entre diferentes SoCs (Inter-Chip).

O \textit{data center} é um ambiente de \textit{clusters} em \textit{rack}, sendo o \textit{cluster} um conjunto de servidores que funcionam como um, ou seja, dois ou mais SoCs se comunicam e se reconhecem como apenas um SoC. Para que haja a comunicação Inter-Chip num sistema de \textit{clusters} em \textit{rack}, é necessário que além do SI entre os processadores de um SoC, haja um SI para os processadores de diferentes SoCs.

%falar de SI entre eles, entrar na IB e Eth. entrar na ENoC

\end{document}
\documentclass[a4paper,12pt]{article}

%	-----------------------------------------------------------------------------	%
%	---------------------------	Pacotes Utilizados	-----------------------------	%
%	-----------------------------------------------------------------------------	%

%	-----------------------------------------------------------------------------	%
%	-------------------------------	Definições	---------------------------------	%
%	-----------------------------------------------------------------------------	%

\setlength{\parindent}{1.3cm} 				% Espaçamento dos paragrafos
\setlength{\parskip}{.2cm} 					% Distância entre paragrafos
\setlength{\oddsidemargin}{0.5cm}   		% Há um offset obrigatorio de
\setlength{\evensidemargin}{0.0cm}   		% 1 inch do lado esquerdo e no
\setlength{\topmargin}{-1.2cm}       		% topo da folha
\setlength{\headsep}{1.0cm}
\setlength{\textwidth}{15.5cm}
\setlength{\textheight}{24.2cm}
\renewcommand{\baselinestretch}{1.2}
\renewcommand{\labelitemi}{\tiny{\textbullet}}
\begin{document}

\begin{center}
\textbf{{\LARGE Introdução}} \\ \vspace{0.5cm}
\end{center}

\textit{Data Center} é um conjunto de recursos físicos (servidores, redes e \textit{storages} e etc.) dedicados que proporciona um alto poder computacional as telecomunicações, processamento, armazenamento e disponibilização de informações. Tais recursos requerem um alto investimento para serem implantados e mantidos, logo é necessário a utilização e otimização ao máximo dos dispositivos adquiridos.

O \textit{data center} é composto de servidores distribuídos em \textit{racks} e a comunicação entre eles ocorre através de cabos e \textit{switches}(que gerência o tráfego de informações entre os dispositivos), que são dispostos num devido local cuja a infraestrutura suporta o funcionamento de todos os equipamentos provendo segurança, energia e controle de temperatura. O objetivo desses recursos estarem num mesmo espaço é trabalhar juntos para aumentar o poder computacional, sendo que cada componente dentro do mesmo pode ser reconhecido como servidor e/ou Sistema em Chip(\textit{System-on-a-Chip}(SoC)).

O SoC é o conjunto de componentes como memória RAM, entrada e saída de dados, disco rígido, processador e etc., num computador ou outro sistema eletrônico e que contenha uma determinada rotina e geralmente um Sistema Operacional (SO) para controlar as mesmas. Os SoCs não estão limitados a computadores e estão presentes em outros dispositivos como smartphones, tablets, mouses e microcontroladores. Quando há mais de um processador num SoC, denomina-se Sistema em Chip com Multiprocessador(\textit{Multi-processor System-on-a-Chip}(MPSoC)), onde há controle de entrada e saída de dados, distribuição dos processos pelo SO para os processadores do MPSoC. Também há administração de acesso a memória, que acontece de maneira compartilhada ou distribuída, ou seja, na memória compartilhada vários processadores acessam a mesma memória, enquanto a distribuída cada um possui um bloco de memória específico. Esses processadores se comunicam através de um Sistema de Intercomunicação (SI) chamado de Rede em Chip(Network-on-Chip(NoC)), sendo que o diálogo entre esses eles para processar algo pode ser tanto entre os do mesmo SoC(Intra-Chip), quanto entre diferentes SoCs (Inter-Chip). Os MPSoCs ajudam tanto na computação paralela quanto na computação distribuída, computação em nuvem e computação de alto desempenho.

O \textit{data center} é um ambiente de \textit{clusters} em \textit{rack}, sendo o \textit{cluster} um conjunto de servidores que funcionam como um, ou seja, dois ou mais MPSoCs se comunicam e se reconhecem como apenas um MPSoC. Para que haja a comunicação Inter-Chip num sistema de \textit{clusters} em \textit{rack}, é necessário que além do SI entre os processadores de um MPSoC, haja um SI para os processadores de diferentes MPSoCs e nesse âmbito há tecnologias consolidadas como Ethernet e InfiniBand. Essas são padrões de rede para comunicação em sistemas Inter \textit{rack} sendo apontadas diversas vezes como SI dos 100 supercomputadores mais poderosos do mundo. Nessa lista o Brazil ocupa a 142ª posição com o computador Fênix da Petrobras, formado por 48.384 núcleos de processamento e InfiniBand como SI, processa algoritmos de simulações e análises geofísicas.
%https://www.techtudo.com.br/noticias/2019/06/conheca-o-fenix-supercomputador-da-petrobras-entre-os-maiores-do-mundo.ghtml

A Ethernet é um conjunto de normas e padrões de rede que define regras numa Rede de Área Local (Local Area Network (LAN)) e intenta trabalhar com tecnologias para garantir a transmissão de dados em alta velocidade sem perda dos mesmos, atuando na camada física e de enlace de dados no modelo Open Systems Interconnection (OSI). Esse protocolo é atualmente padronizado pelo IEEE 802.3, um grupo de estudo pertencente ao Institute of Electrical and Electronics Engineers (IEEE). Já a InfiniBand (IB) é uma rede padronizada pela InfiniBand Trade Association destinada para computação numa LAN, provendo um fácil meio para transporte de mensagens direto de uma aplicação a outra aplicação, storage ou sistema operacional. Enquanto a Ethernet foca na transmissão de bits de dados numa rede, a IB visa criar um canal direto de comunicação, numa rede, entre elementos de uma aplicação sem necessidade de intervenção do sistema operacional.

Todos esses aspectos estão no ambiente de \textit{clusters} em \textit{rack} e ao considerar a computação de alto desempenho, a demanda da computação e poder de processamento é tão crescente quanto possível, dessa forma prover uma alternativa que ofereça a expansão desses recursos de forma escalável e veloz é um tema muito abordado nas pesquisas cotidianas.

%Referenciar (Grun & InfiniBand® Trade Association, 2010).


\end{document}
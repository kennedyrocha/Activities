\documentclass[a4paper,12pt]{article}

\setlength{\parindent}{1.2cm}
\setlength{\parskip}{.2cm}
\setlength{\oddsidemargin}{0.5cm}    % Há um offset obrigatorio de
\setlength{\evensidemargin}{0.0cm}   % 1 inch do lado esquerdo e no
\setlength{\topmargin}{-1.2cm}       % topo da folha
\setlength{\headsep}{1.0cm}
\setlength{\textwidth}{15.5cm}
\setlength{\textheight}{24.2cm}
\renewcommand{\baselinestretch}{1.2}
\renewcommand{\labelitemi}{\tiny{\textbullet}}
\begin{document}

\begin{center}
\textbf{{\LARGE Introdução}} \\ \vspace{0.5cm}
\end{center}

\textit{Data Center} é um conjunto de recursos físicos (servidores, redes e \textit{storages}) dedicados que proporciona um alto poder computacional as telecomunicações, processamento, armazenamento e disponibilização de informações. Tais recursos requerem um alto investimento para serem implantados e mantidos e, para tantos gastos, é necessário a utilizão e otimização ao máximo dos dispositivos adiquiridos.

O \textit{data center} é composto de servidores distribuídos em \textit{racks} e a comunicação entre eles ocorre através de cabos e \textit{switches}(que gerência o tráfego de informações entre os dispositivos no mesmo), que são dispostos num devido local cuja a infraestrutura suporta o funcionamento de todos os equipamentos, provendo segurança, energia e controle de temperatura. O objetivo desses recursos estarem num mesmo espaço é trabalhar juntos para aumentar o poder computacional, sendo que cada componente dentro do mesmo pode ser reconhecido como servidor e/ou Sistema em Chip(\textit{System-on-a-Chip}(SoC)).

O SoC é o conjunto de componentes num computador(memória RAM, disco dígido e processador e etc.), ou outro sistema eletrônico e, que contenha uma determinada rotina e geralmente um Sistema Operacional(SO) para controlar as mesmas. Quando há mais de um processador dentro do SoC, denominado-se de Sistema em Chip com Multiprocessador(\textit{Multi-processor System-on-a-Chip}(MPSoC)), onde há uma hierarquia para acesso a memória, controle de entrada e saída de dados e distribuíção dos processos pelo SO, de uma ou várias aplicações, para os processadores do MPSoC, que se comunicam através de um Sistema de Intercomunicação(SI) chamado de Rede em Chip(Network-on-Chip(NoC)).

\end{document}
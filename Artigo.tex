\documentclass[a4paper,12pt]{article}

\setlength{\parindent}{1.2cm}
\setlength{\parskip}{.2cm}
\setlength{\oddsidemargin}{0.5cm}    % Há um offset obrigatorio de
\setlength{\evensidemargin}{0.0cm}   % 1 inch do lado esquerdo e no
\setlength{\topmargin}{-1.2cm}       % topo da folha
\setlength{\headsep}{1.0cm}
\setlength{\textwidth}{15.5cm}
\setlength{\textheight}{24.2cm}
\renewcommand{\baselinestretch}{1.2}
\renewcommand{\labelitemi}{\tiny{\textbullet}}
\begin{document}

\begin{center}
\textbf{{\LARGE Introdução}} \\ \vspace{0.5cm}
\end{center}

\textit{Data Center} é um conjunto de recursos físicos (servdores, redes e armazenamento) dedicados que proporciona um alto poder computacional as telecomunicações, processamento, armazenamento e disponibilização de informações. É composto de servidores distribuídos em \textit{racks} e a comunicação entre eles ocorre através de cabos e \textit{switches}(que gerência o tráfego de informações entre os dispositivos no \textit{data center}), que são dispostos num devido local cuja a infraestrutura suporta o funcionamento de todos os equipamentos, provendo segurança, energia e controle de temperatura.


\end{document}